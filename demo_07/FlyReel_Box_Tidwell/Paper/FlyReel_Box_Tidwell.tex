\documentclass[11pt]{paper}
\usepackage{fullpage}
\usepackage{palatino}
\usepackage{amsfonts,amsmath,amssymb}
% \usepackage{graphicx}

\usepackage{listings}
\usepackage{textcomp}
\usepackage{color}

\definecolor{dkgreen}{rgb}{0,0.6,0}
\definecolor{gray}{rgb}{0.5,0.5,0.5}
\definecolor{mauve}{rgb}{0.58,0,0.82}

\lstset{frame=tb,
  language=R,
  aboveskip=3mm,
  belowskip=3mm,
  showstringspaces=false,
  columns=flexible,
  basicstyle={\small\ttfamily},
  numbers=none,
  numberstyle=\tiny\color{gray},
  keywordstyle=\color{blue},
  commentstyle=\color{dkgreen},
  stringstyle=\color{mauve},
  breaklines=true,
  breakatwhitespace=true,
  tabsize=3
}



\ifx\pdftexversion\undefined
    \usepackage[dvips]{graphicx}
\else
    \usepackage[pdftex]{graphicx}
    \usepackage{epstopdf}
    \epstopdfsetup{suffix=}
\fi

\usepackage{subfig}



% Trying different tips from Google help to get a summary to print.
% \usepackage[T1]{fontenc} 
% \usepackage[utf8]{inputenc}
% 
% \usepackage[utf8]{inputenc}
% \usepackage[utf8x]{inputenc}
\UseRawInputEncoding
% This allows pdflatex to print the curly quotes in the
% significance codes in the output of the GAM.


\begin{document}

%%%%%%%%%%%%%%%%%%%%%%%%%%%%%%%%%%%%%%%%
% Problem Set 7
%%%%%%%%%%%%%%%%%%%%%%%%%%%%%%%%%%%%%%%%

\pagestyle{empty}
{\noindent\bf Spring 2023 \hfill Firstname M.~Lastname}
\vskip 16pt
\centerline{\bf University of Central Florida}
\centerline{\bf College of Business}
\vskip 16pt
\centerline{\bf QMB 6911}
\centerline{\bf Capstone Project in Business Analytics}
\vskip 10pt
\centerline{\bf Solutions:  Problem Set \#7}
\vskip 32pt
\noindent
% 
% 
\section{Data Description}
% 
By engaging an industry consultant to gather relevant and appropriate 
information, your firm has been able to put together data concerning 248 
different fly-fishing reels, over one-half of which are produced in the 
United States, with the remainder being produced in Asia---either in China 
or Korea.  These data are contained in the file {\tt FlyReels.csv}, which is
available in the {\tt Data} folder.
Each fly-fishing reel in the data set is a row, while the columns correspond 
to the variables whose names and definitions are the following:
\bigskip
\begin{table}[ht]
\centering
\begin{tabular}{ll}
  \hline
    Variable & Definition \\
  \hline

    {\tt Name}        &product name (a string) \\ 
    {\tt Brand}       &brand name (a string) \\ 
    {\tt Weight}      &weight of reel in ounces (a real number) \\ 
    {\tt Diameter}    &diameter of reel in inches (a real number) \\ 
    {\tt Width}       &width of reel in inches (a real number) \\ 
    {\tt Price}       &price of reel in dollars (a real number) \\ 
    {\tt Sealed}      &whether the reel is sealed; {\tt "Yes"} versus
                        {\tt "No"} (a string) \\ 
    {\tt Country}     &country of manufacture, (a string) \\ 
    {\tt Machined}    &whether the reel is machined versus cast;
                        machined={\tt "Yes"}, \\ 
                      &while cast={\tt "No"} (a string) \\ 
  \hline
\end{tabular}
%\caption{Summary of Numeric Variables}
%\label{tab:summary}
\end{table}


I will revisit the recommended linear model
from Problem Set \#6. 
%
In doing so, I will investigate any nonlinear relationships
by incorporating a nonlinear but parametric specification
for the value of the dimensions of the reels:
the width, diameter, and density, 
which constitute the continuous variables in the dataset.
This parametric analysis will be performed
using the Box-Tidwell framework
to investigate whether the value of these characteristics
are best described with parametric nonlinear forms. 


%%%%%%%%%%%%%%%%%%%%%%%%%%%%%%%%%%%%%%%%
\clearpage
\section{Linear Regression Model}
%%%%%%%%%%%%%%%%%%%%%%%%%%%%%%%%%%%%%%%%

A natural staring point is the recommended linear model
from Problem Set \#6. 

\subsection{Linear Model with \texttt{Sealed*Made\_in\_USA} Interaction}

Last week I investigated whether 
the functional form should include different specifications by
country of manufacture.
% 
The model included the continuous variables 
width, diameter, and density, 
as well as categorical variables for 
country of manufacture, 
and whether or not the reels were sealed or machined. 
% 
In addition to the indicator for the country of manufacture, the model included an indicator for an interaction between
the the country of manufacture indicator and the indicator for whether the reels were sealed or unsealed. 
% 
The dependent variable was chosen as 
the logarithm of the fly reel price, 
since the results were similar to those from the model 
with the optimal Box-Cox transformation, 
without the added complexity. 
% 
The results of this regression specification are shown in 
Table \ref{tab:reg_sealed_USA}. 
% 

\begin{table}
\begin{center}
\begin{tabular}{l c}
\hline
 & Model 1 \\
\hline
(Intercept)                 & $2.00999^{***}$ \\
                            & $(0.26125)$     \\
Width                       & $0.33575^{*}$   \\
                            & $(0.15622)$     \\
Diameter                    & $0.39567^{***}$ \\
                            & $(0.05076)$     \\
Density                     & $1.21296^{***}$ \\
                            & $(0.21948)$     \\
SealedYes                   & $0.62731^{***}$ \\
                            & $(0.08622)$     \\
MachinedYes                 & $0.64934^{***}$ \\
                            & $(0.08320)$     \\
made\_in\_USATRUE           & $0.74633^{***}$ \\
                            & $(0.09247)$     \\
SealedYes:made\_in\_USATRUE & $-0.29519^{**}$ \\
                            & $(0.10092)$     \\
\hline
R$^2$                       & $0.74893$       \\
Adj. R$^2$                  & $0.74160$       \\
Num. obs.                   & $248$           \\
\hline
\multicolumn{2}{l}{\scriptsize{$^{***}p<0.001$; $^{**}p<0.01$; $^{*}p<0.05$}}
\end{tabular}
\caption{Linear Model for Fly Reel Prices}
\label{tab:reg_sealed_USA}
\end{center}
\end{table}

% 
Next, I will attempt to improve on this specification
by investigating the potential for nonlinear functional forms. 








%%%%%%%%%%%%%%%%%%%%%%%%%%%%%%%%%%%%%%%%
\clearpage
\section{Nonlinear Specifications}
%%%%%%%%%%%%%%%%%%%%%%%%%%%%%%%%%%%%%%%%




% \pagebreak
\subsection{The Box--Tidwell Transformation}

The Box--Tidwell function tests for non-linear relationships
to the mean of the dependent variable.
The nonlinearity is in the form of an
exponential transformation in the form of the Box-Cox
transformation, except that the transformation is taken
on the explanatory variables.


\subsubsection{Transformation of Width}


Performing the transformation on the 
width variable
produces a modified form of the linear model.
This specification allows a single exponential
transformation on 
width, 
rather than a linear form.

\begin{verbatim} MLE of lambda Score Statistic (z) Pr(>|z|)
        1.1615              0.1587   0.8739

iterations =  5 
\end{verbatim}

The \textsf{R} output is the statistics for a test of nonlinearity:
that the exponent $\lambda$ in the Box--Tidwell transformation is zero.
%
The "\texttt{MLE of lambda}" statistic is the optimal exponent on horsepower.
Similar to the Box-Cox transformation,
with Box-Tidwell, the exponents are on the explanatory variables
and are all called lambda, in contrast
to the parameter $\tau$ in our class notes.
%  
The exponent is not significantly different from one. 
This supports a linear form for Width,
confirming our result from the nonparametric analysis.



\subsubsection{Transformation of Diameter}


\begin{verbatim} MLE of lambda Score Statistic (z) Pr(>|z|)
      -0.35213             -1.2106    0.226

iterations =  4 
\end{verbatim}


This is very weak evidence for
an inverse square root transformation
for diameter but it is not estimated accurately.
%
There is no evidence for a transformation 
with a coefficient different from 1, 
which suggests
a purely linear relationship between \texttt{log\_Price}
and 
diameter.
Next, I will consider the possibility of nonlinearity 
in the value of the density of a reel. 

\subsubsection{Transformation of Density}


\begin{verbatim} MLE of lambda Score Statistic (z) Pr(>|z|)
       0.19315             -1.3448   0.1787

iterations =  10 
\end{verbatim}

Similar to Diameter, this is very weak evidence for
a square root transformation
for diameter but it is not estimated accurately.
%
Conclude that the linear model is the best choice, 
when nonlinear options are restricted to this parametric form.
It is possible, however, that a nonlinear relationship exists
but takes on a form not captured by this specific framework. 





%%%%%%%%%%%%%%%%%%%%%%%%%%%%%%%%%%%%%%%%
\end{document}
%%%%%%%%%%%%%%%%%%%%%%%%%%%%%%%%%%%%%%%%
