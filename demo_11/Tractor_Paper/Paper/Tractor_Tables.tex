%%%%%%%%%%%%%%%%%%%%%%%%%%%%%%%%%%%%%%%%%%%%%%%%%%%%%%%%%%%%
%
%\documentclass[11pt]{article}
%\usepackage{fullpage}
%\begin{document}
%
%
%%%%%%%%%%%%%%%%%%%%%%%%%%%%%%%%%%%%%%%%%%%%%%%%%%%%%%%%%%%%%
%
%{\noindent\bf Spring 2023 \hfill First Last}
%\vskip 16pt
%\centerline{\bf University of Central Florida}
%\centerline{\bf College of Business }
%\vskip 16pt
%\centerline{\bf QMB 6912}
%\centerline{\bf Capstone Project in Business Analytics}
%\vskip 10pt
%\centerline{\bf Problem Set \#4}
%\vskip 32pt
%\noindent
%% 
%\section{Data Description}
%
%This analysis follows the script \texttt{Tractor\_Tables.R}. 
%The script tabulates various statistics from the variables that may be used to estimate a model for used tractor prices.  
%It uses the data from \texttt{TRACTOR7.csv} in the \texttt{Data} folder. 
%The dataset includes the following variables.
%\begin{table}[h!]
%\begin{tabular}{l l l}
%
%$saleprice_i$ & = & the price paid for tractor $i$ in dollars \\
%% 
%$horsepower_i$ & = & the horsepower of tractor $i$ \\
%$age_i$ & = & the number of years since tractor $i$ was manufactured  \\
%$enghours_i$ & = & the number of hours of use recorded for tractor $i$  \\
%$diesel_i$ & = & an indicator of whether tractor $i$ runs on diesel fuel \\ %, $0$ otherwise \\
%$fwd_i$ & = & an indicator of whether tractor $i$ has four-wheel drive \\ %, $0$ otherwise \\
%$manual_i$ & = & an indicator of whether tractor $i$ has a manual transmission \\ %, $0$ otherwise \\
%$johndeere_i$ & = & an indicator of whether tractor $i$ is manufactured by John Deere \\ %, $0$ otherwise \\
%$cab_i$ & = & an indicator of whether tractor $i$ has an enclosed cab \\ %, $0$ otherwise \\
%% 
%$spring_i$ & = & an indicator of whether tractor $i$ was sold in April or May \\ %, $0$ otherwise \\
%$summer_i$ & = & an indicator of whether tractor $i$ was sold between June and September \\ %, $0$ otherwise \\
%$winter_i$ & = & an indicator of whether tractor $i$ was sold between December and March \\ %, $0$ otherwise \\
%
%\end{tabular}
%\end{table}
%%

\bigskip
\noindent
I have downloaded the file {\tt TRACTOR7.csv}, 
loaded the data described above into 
\textsf{R}, 
calculated the summary statistics for these data, 
and finally, presented 
these statistics in \LaTeX\ tables.
These operations are all performed by the script 
{\tt Tractor\_Tables.R} in the {\tt Code} folder. 
The script uses an \textsf{R} package called {\tt xtable} 
to automate the
production of the tables from a data frames in \textsf{R}.)

\medskip
\noindent
I analyze the data in subsets, according to the makes of tractors, 
calculating the summary statistics for each subset and present these 
statistics in the \LaTeX\ tables that follow.

\vfill

%%%%%%%%%%%%%%%%%%%%%%%%%%%%%%%%%%%%%%%%%%%%%%%%%%%%%%%%%%%%

\pagebreak
\section{Summary by Make of Tractors}

Table \ref{tab:summ_by_make} 
lists summary statistics for numeric variables
in separate rows for subsamples defined by the make of tractors, 
according to whether they were manufactured by John Deere. 

% latex table generated in R 4.0.5 by xtable 1.8-4 package
% Tue Feb 07 16:01:18 2023
\begin{table}[ht]
\centering
\begin{tabular}{rll}
  \hline
 & John Deere & Other \\ 
  \hline
Min. saleprice &   2800 &   1500 \\ 
  Mean saleprice &  27973 &  19557 \\ 
  Max. saleprice & 200000 & 135000 \\ 
  Min. horsepower &  16.00 &  18.0 \\ 
  Mean horsepower &  99.13 & 101.4 \\ 
  Max. horsepower & 491.00 & 535.0 \\ 
  Min. age &  2.00 &  2.00 \\ 
  Mean age & 18.03 & 15.56 \\ 
  Max. age & 33.00 & 33.00 \\ 
  Min. enghours &    22 &     1 \\ 
  Mean enghours &  4023 &  3449 \\ 
  Max. enghours & 10734 & 18744 \\ 
   \hline
\end{tabular}
\caption{Summary by Make of Tractor} 
\label{tab:summ_by_make}
\end{table}



\pagebreak
\section{Indicator Variables by Make of Tractors}

Table \ref{tab:ind_by_make} lists the frequencies of observations of 
each brand of tractor across a series of indicator variables, 
which includes whether a tractor
has a diesel engine, four-wheel-drive transmission, a manual transmission, 
or an enclosed cab to protect the operator.

% latex table generated in R 4.2.2 by xtable 1.8-4 package
% Wed Feb  8 20:59:56 2023
\begin{table}[ht]
\centering
\begin{tabular}{rrrr}
  \hline
 & John Deere & Other & Totals \\ 
  \hline
Total & 39 & 237 & 276 \\ 
  Gasoline & 9 & 16 & 25 \\ 
  Diesel & 30 & 221 & 251 \\ 
  2WD & 19 & 101 & 120 \\ 
  4WD & 20 & 136 & 156 \\ 
  Automatic & 9 & 73 & 82 \\ 
  Manual & 30 & 164 & 194 \\ 
  No Cab & 25 & 101 & 126 \\ 
  Has Cab & 14 & 136 & 150 \\ 
   \hline
\end{tabular}
\caption{Indicator Variables by Make of Tractor} 
\label{tab:ind_by_make}
\end{table}





\pagebreak
\section{Season Sold by Make of Tractor}

Table \ref{tab:season_sold_by_make} lists the frequencies of observations of 
sales of the brand categories of tractors 
over the four seasons. 


% latex table generated in R 4.1.1 by xtable 1.8-4 package
% Tue Jan 10 15:45:22 2023
\begin{table}[ht]
\centering
\begin{tabular}{rrrr}
  \hline
 & John Deere & Other & Totals \\ 
  \hline
Fall & 14 & 89 & 103 \\ 
  Spring & 9 & 53 & 62 \\ 
  Summer & 6 & 58 & 64 \\ 
  Winter & 10 & 37 & 47 \\ 
  Totals & 39 & 237 & 276 \\ 
   \hline
\end{tabular}
\caption{Season Sold by Make of Tractor} 
\label{tab:season_sold_by_make}
\end{table}




%%%%%%%%%%%%%%%%%%%%%%%%%%%%%%%%%%%%%%%%%%%%%%%%%%%%%%%%%%%%

% \pagebreak
\section{Average Price of Tractors by Season Sold}

Table \ref{tab:avg_price_by_season_sold} lists the 
average price of tractors sold
over the four seasons. 

% latex table generated in R 4.2.2 by xtable 1.8-4 package
% Wed Feb  8 21:31:03 2023
\begin{table}[ht]
\centering
\begin{tabular}{rrr}
  \hline
 & John Deere & Other \\ 
  \hline
Fall & 35875 & 16574 \\ 
  Spring & 37483 & 23958 \\ 
  Summer & 11908 & 16062 \\ 
  Winter & 17990 & 25909 \\ 
   \hline
\end{tabular}
\caption{Average Price of Tractors by Season Sold} 
\label{tab:avg_price_by_season_sold}
\end{table}




%%%%%%%%%%%%%%%%%%%%%%%%%%%%%%%%%%%%%%%%%%%%%%%%%%%%%%%%%%%%


\pagebreak
\section{Correlation Matrices}
To plot the correlation between variables, 
we use the logarithm of prices, which had a better-behaved distribution, 
according to our results in Problem Set \#3.
We separate the analysis into numerical and indicator variables 


Table \ref{tab:correlation_num} shows the correlation 
between the log.~of tractor prices
and the numeric variables horsepower, age and engine hours.


% latex table generated in R 4.0.5 by xtable 1.8-4 package
% Tue Feb 07 15:58:03 2023
\begin{table}[ht]
\centering
\begin{tabular}{rrrrr}
  \hline
 & Price & Weight & Diameter & Width \\ 
  \hline
Price & 1.000 & 0.546 & 0.498 & 0.254 \\ 
  Weight & 0.546 & 1.000 & 0.833 & 0.687 \\ 
  Diameter & 0.498 & 0.833 & 1.000 & 0.643 \\ 
  Width & 0.254 & 0.687 & 0.643 & 1.000 \\ 
   \hline
\end{tabular}
\caption{Correlation Matrix of Prices and Numeric Variables} 
\label{tab:correlation_num}
\end{table}



Table \ref{tab:correlation_ind} shows the correlation 
between the log.~of tractor prices
and the indicators for whether a tractor
has a diesel engine, four-wheel-drive transmission, a manual transmission, 
or an enclosed cab to protect the operator. 


% latex table generated in R 4.2.2 by xtable 1.8-4 package
% Wed Feb  8 21:13:33 2023
\begin{table}[ht]
\centering
\begin{tabular}{rrrrrr}
  \hline
 & Log. of Price & Diesel & FWD & Manual & Cab \\ 
  \hline
Log. of Price & 1.000 & 0.331 & 0.522 & 0.039 & 0.588 \\ 
  Diesel & 0.331 & 1.000 & 0.207 & 0.347 & 0.344 \\ 
  FWD & 0.522 & 0.207 & 1.000 & 0.006 & 0.223 \\ 
  Manual & 0.039 & 0.347 & 0.006 & 1.000 & 0.280 \\ 
  Cab & 0.588 & 0.344 & 0.223 & 0.280 & 1.000 \\ 
   \hline
\end{tabular}
\caption{Correlation Matrix of Log. Prices and Indicator Variables} 
\label{tab:correlation_ind}
\end{table}



%%%%%%%%%%%%%%%%%%%%%%%%%%%%%%%%%%%%%%%%%%%%%%%%%%%%%%%%%%%%


% \end{document}

%%%%%%%%%%%%%%%%%%%%%%%%%%%%%%%%%%%%%%%%%%%%%%%%%%%%%%%%%%%%
