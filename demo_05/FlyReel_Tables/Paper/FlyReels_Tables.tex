%%%%%%%%%%%%%%%%%%%%%%%%%%%%%%%%%%%%%%%%%%%%%%%%%%%%%%%%%%%

\documentclass[11pt]{article}
\usepackage{fullpage}
\begin{document}


%%%%%%%%%%%%%%%%%%%%%%%%%%%%%%%%%%%%%%%%%%%%%%%%%%%%%%%%%%%%

{\noindent\bf Spring 2021 \hfill First Last}
\vskip 16pt
\centerline{\bf University of Central Florida}
\centerline{\bf College of Business }
\vskip 16pt
\centerline{\bf QMB 6912}
\centerline{\bf Capstone Project in Business Analytics}
\vskip 10pt
\centerline{\bf Problem Set \#3}
\vskip 32pt
\noindent
% 
\section{Data Description}
% 
By engaging an industry consultant to gather relevant and appropriate 
information, your firm has been able to put together data concerning 248 
different fly-fishing reels, over one-half of which are produced in the 
United States, with the remainder being produced in Asia---either in China 
or Korea.  These data are contained in the file {\tt FlyReels.csv}, which is
available in the {\tt Data} folder.
Each fly-fishing reel in the data set is a row, while the columns correspond 
to the variables whose names and definitions are the following:
\bigskip
\begin{table}[ht]
\centering
\begin{tabular}{ll}
  \hline
    Variable & Definition \\
  \hline

    {\tt Name}        &product name (a string) \\ 
    {\tt Brand}       &brand name (a string) \\ 
    {\tt Weight}      &weight of reel in ounces (a real number) \\ 
    {\tt Diameter}    &diameter of reel in inches (a real number) \\ 
    {\tt Width}       &width of reel in inches (a real number) \\ 
    {\tt Price}       &price of reel in dollars (a real number) \\ 
    {\tt Sealed}      &whether the reel is sealed; {\tt "Yes"} versus
                        {\tt "No"} (a string) \\ 
    {\tt Country}     &country of manufacture, (a string) \\ 
    {\tt Machined}    &whether the reel is machined versus cast;
                        machined={\tt "Yes"}, \\ 
                      &while cast={\tt "No"} (a string) \\ 
  \hline
\end{tabular}
%\caption{Summary of Numeric Variables}
%\label{tab:summary}
\end{table}

\bigskip
\noindent
I have downloaded the file {\tt FlyReels.csv}, 
loaded the data described above into 
\textsf{R}, 
calculated the summary statistics for these data, 
and finally, presented 
these statistics in \LaTeX\ tables.
These operations are all performed by the script 
{\tt FlyReel\_Tables.R} in the {\tt Code} folder. 
The script uses an \textsf{R} package called {\tt xtable} 
to automate the
production of the tables from a data frames in \textsf{R}.)

\medskip
\noindent
I analyze the data in subsets, according to country of manufacture, 
calculating the summary statistics for each subset and present these 
statistics in the \LaTeX\ tables that follow.

\vfill

%%%%%%%%%%%%%%%%%%%%%%%%%%%%%%%%%%%%%%%%%%%%%%%%%%%%%%%%%%%%

\pagebreak
\section{Summary by Country of Manufacture}

Table \ref{tab:summ_by_country} lists summary statistics for numeric variables
in separate columns for subsamples defined by the country of manufacture. 

% latex table generated in R 4.0.5 by xtable 1.8-4 package
% Thu Jan 27 21:16:18 2022
\begin{table}[ht]
\centering
\begin{tabular}{rlll}
  \hline
 & China & Korea & USA \\ 
  \hline
Min. Weight &  3.000 &  7.296 & 12.900 \\ 
  Mean Weight &  2.100 &  6.500 & 15.097 \\ 
  Max. Weight &  2.540 &  6.459 & 14.800 \\ 
  Min. Diameter & 2.500 & 3.935 & 5.250 \\ 
  Mean Diameter & 2.750 & 3.925 & 5.500 \\ 
  Max. Diameter & 2.700 & 3.878 & 5.430 \\ 
  Min. Width & 0.790 & 1.093 & 1.570 \\ 
  Mean Width & 0.7874 & 1.1434 & 1.5800 \\ 
  Max. Width & 0.750 & 1.070 & 1.688 \\ 
  Min. Price & 129.0 & 331.6 & 600.0 \\ 
  Mean Price &  34.99 & 280.22 & 839.00 \\ 
  Max. Price &  200.0 &  484.9 & 1095.0 \\ 
   \hline
\end{tabular}
\caption{Summary by Country of Manufacture} 
\label{tab:summ_by_country}
\end{table}



\pagebreak
\section{Country of Manufacture by Brand of Fly Reel}

Table \ref{tab:country_by_brand} lists the frequencies of observations of 
each brand of fly reel by country of manufacture. 

% latex table generated in R 4.0.5 by xtable 1.8-4 package
% Thu Jan 27 21:16:18 2022
\begin{table}[ht]
\centering
\begin{tabular}{rrrrr}
  \hline
 & China & Korea & USA & Total \\ 
  \hline
3-TAND & 15 & 0 & 0 & 15 \\ 
  Abel & 0 & 0 & 15 & 15 \\ 
  Allen & 0 & 18 & 7 & 25 \\ 
  Aspen & 0 & 0 & 8 & 8 \\ 
  Bauer & 0 & 0 & 2 & 2 \\ 
  Cheeky & 11 & 0 & 0 & 11 \\ 
  ECHO & 0 & 12 & 0 & 12 \\ 
  Galvan & 0 & 0 & 23 & 23 \\ 
  Hatch & 0 & 0 & 8 & 8 \\ 
  Loop & 0 & 14 & 0 & 14 \\ 
  Nautilus & 0 & 0 & 15 & 15 \\ 
  Orvis & 1 & 0 & 1 & 2 \\ 
  Ross & 0 & 0 & 28 & 28 \\ 
  Sage & 0 & 6 & 0 & 6 \\ 
  Taylor & 0 & 12 & 0 & 12 \\ 
  TFO & 0 & 16 & 0 & 16 \\ 
  Tibor & 0 & 0 & 4 & 4 \\ 
  Waterworks-Lamson & 0 & 8 & 24 & 32 \\ 
  Totals & 27 & 86 & 135 & 248 \\ 
   \hline
\end{tabular}
\caption{Country of Manufacture by Brand of Fly Reel} 
\label{tab:country_by_brand}
\end{table}


\pagebreak
\section{Reel Design by Brand of Fly Reel}

Table \ref{tab:design_by_brand} lists the frequencies of observations of 
each brand of fly reel across two categorical variables:
whether the reel is sealed
and whether the reel is machined versus cast. 


% latex table generated in R 4.1.1 by xtable 1.8-4 package
% Sun Apr 10 13:32:52 2022
\begin{table}[ht]
\centering
\begin{tabular}{rrrrrr}
  \hline
 & Unsealed & Sealed & Cast & Machined & Total \\ 
  \hline
3-TAND & 0 & 15 & 0 & 15 & 15 \\ 
  Abel & 9 & 6 & 0 & 15 & 15 \\ 
  Allen & 8 & 17 & 1 & 24 & 25 \\ 
  Aspen & 8 & 0 & 0 & 8 & 8 \\ 
  Bauer & 0 & 2 & 0 & 2 & 2 \\ 
  Cheeky & 6 & 5 & 6 & 5 & 11 \\ 
  ECHO & 9 & 3 & 12 & 0 & 12 \\ 
  Galvan & 20 & 3 & 0 & 23 & 23 \\ 
  Hatch & 0 & 8 & 0 & 8 & 8 \\ 
  Loop & 0 & 14 & 0 & 14 & 14 \\ 
  Nautilus & 4 & 11 & 0 & 15 & 15 \\ 
  Orvis & 0 & 2 & 0 & 2 & 2 \\ 
  Ross & 21 & 7 & 0 & 28 & 28 \\ 
  Sage & 2 & 4 & 0 & 6 & 6 \\ 
  Taylor & 0 & 12 & 0 & 12 & 12 \\ 
  TFO & 4 & 12 & 4 & 12 & 16 \\ 
  Tibor & 3 & 1 & 0 & 4 & 4 \\ 
  Waterworks-Lamson & 0 & 32 & 8 & 24 & 32 \\ 
  Totals & 94 & 154 & 31 & 217 & 248 \\ 
   \hline
\end{tabular}
\caption{Reel Design by Brand of Fly Reel} 
\label{tab:design_by_brand}
\end{table}



%%%%%%%%%%%%%%%%%%%%%%%%%%%%%%%%%%%%%%%%%%%%%%%%%%%%%%%%%%%%


\end{document}

%%%%%%%%%%%%%%%%%%%%%%%%%%%%%%%%%%%%%%%%%%%%%%%%%%%%%%%%%%%%
