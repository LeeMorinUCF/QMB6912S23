

%%%%%%%%%%%%%%%%%%%%%%%%%%%%%%%%%%%%%%%%%%%%%%%%%%%%%%%%%%%

\documentclass[11pt]{article}
\usepackage{fullpage}
\begin{document}


%%%%%%%%%%%%%%%%%%%%%%%%%%%%%%%%%%%%%%%%%%%%%%%%%%%%%%%%%%%%

% \Xpoint
% \nopagenumbers
% \autoeqnooff
% \input LaTeX.tex
{\noindent\bf Spring 2023 \hfill Lealand Morin}
\vskip 16pt
\centerline{\bf University of Central Florida}
\centerline{\bf College of Business }
\vskip 16pt
\centerline{\bf QMB 6912}
\centerline{\bf Capstone Project in Business Analytics}
\vskip 10pt
\centerline{\bf Problem Set \#2}
\vskip 10pt
\centerline{\bf Due Date: Sunday, 22 January 2023, at 11:59 PM.}
\vskip 32pt
\noindent




A data engineer for several realtors has gathered relevant and appropriate 
information, and organized a dataset concerning 1,862 sales of detached houses.
The data engineer scraped data from many real estate Websites
and combined it with the records from the realtors to identify
the houses that were available for rent during the year after each sale.
Around one third of the houses were identified as rental units, 
while the other two thirds were not observed for rent
and were presumably occupied by the owner.
% 
These data are contained in the file {\tt HomeSales.dat}, which is
available in the {\tt Data} folder of the course repository
and on the course Webpage under Module 2.
% 

Each house in the dataset is a row, while the columns correspond 
to the variables whose names and definitions are the following:

\bigskip
\begin{table}[ht]
\centering
\begin{tabular}{ll}
  \hline
    Variable & Definition \\
  \hline
    {\tt year\_built}      & the year in which the house was constructed \cr
    {\tt num\_beds}      & the number of bedrooms in the house \cr
    {\tt num\_baths}      & the number of bathrooms in the house\cr
    {\tt floor\_space}      & the area of floor space in the house, in square feet\cr
    {\tt lot\_size}      & the area of lot on which the house was built, in square feet \cr
    {\tt has\_garage}      & an indicator for whether the house has a garage \cr
    {\tt has\_encl\_patio}      & an indicator for whether the house has an enclosed patio \cr
    {\tt has\_security\_gate}      & an indicator for whether the property is accessed through a security gate \cr
    {\tt has\_pool}      & an indicator for whether the property includes a pool \cr
    {\tt transit\_score}      & an integer to represent the convenience of transportation options \cr
    {\tt school\_score}      & an integer to represent the quality of the schools in the county  \cr
    {\tt type\_of\_buyer}      & a categorical variable to indicate the type of buyer, \cr
		& either ``Owner-Occupied'' or ``Rental''  \cr
    {\tt price}      & the price at which the home was sold \cr
  \hline
\end{tabular}
\end{table}

\bigskip
\noindent

% \vfill


\bigskip
The first several variables are self-explanatory 
but some of the variables above warrant some description.
The last two integers, \texttt{transit\_score} and \texttt{school\_score}
describe the amenities available to the residents in each house. 
The {\tt transit\_score} is an integer from one to ten that indicates the 
convenience of transportation options at the location of the house. 
It factors in the availability of public transportation options, 
as well as the proximity to business districts and major highways, 
with a higher score indicating a higher level of convenience. 
Similarly the {\tt school\_score} is an integer from one to ten that indicates the 
quality of the schools in the county. 
It is compiled from indicators of college attendance, test scores, 
as well as other measures of school and student performance.
As with the {\tt transit\_score}, a higher {\tt school\_score}
indicates better school quality. 
The variables listed above also include the dependent variable. 
The {\tt price} variable is the price at which the home was sold. 



\medskip
Begin your analysis by focusing on the dependent variable, {\tt price},
including all observations together, regardless of 
whether the home is owner-occupied or used as a rental property.
Calculate the relative histogram of home prices and plot a graph.
Then, calculate the kernel-smoothed probability density function of 
home prices. 
Choose an appropriate bandwidth that strikes a balance between
smoothness, for ease of interpretation, and flexibility, for fitting the data accurately. 
Now graph these objects in \textsf{R} and output the relevant graphs in a format 
that can be loaded into \LaTeX\ for further processing.


\medskip
Repeat the above procedures using 
the natural logarithm of home prices.




\medskip
Next, recalculate the density of sale prices,
this time disaggregating by type of owner.
That is, create separate kernel-smoothed probability density functions
for homes sold as rental properties and homes occupied by the owner. 
Note that you might find that different bandwidths
are appropriate for both subsamples, 
so you might conduct the analysis separately for each subsample. 
Next, depict these two objects on the same graph, labeling them appropriately
so the reader can distinguish among them.


\medskip
Again, repeat the previous steps
for the separate samples by type of owner using 
the natural logarithm of home prices as the dependent variable.


\medskip
Describe any differences that you find between the densities by type of owner.
Do the prices follow a different distribution for each type?
Would you expect to find different distributions of the characteristics of homes?
Would you expect to find different relationships between the characteristics of homes
and their sale prices?
We will investigate your conjectures through the next several problem sets.






\pagebreak
Prepare and compile your work in \LaTeX\
and include scripts for any of the 
calculations in \textsf{R}.
In particular, create the following directory structure, separate from your
existing work:

\begin{itemize}
\item {\tt Code/}
\item {\tt Data/}
\item {\tt Figures/}
%\item {\tt Tables/}
%\item {\tt Text/}
\item {\tt Paper/}
\item {\tt Misc/}
\end{itemize}

\bigskip\noindent
In a file called {\tt README.md}, 
which should also live in the directory
containing the above folders, 
provide the instructions concerning how to run the 
executable shell script {\tt DoWork.sh} (in the same directory) 
that will execute the code that 
produced all of the answers collected and documented in 
your report,
which will live in the subdirectory {\tt Paper/}.
In the subdirectory {\tt Code/}, keep the {\textsf R} code; 
in {\tt Data/} keep the raw data file you downloaded, so that {\tt DoWork.sh} 
can load it into {\textsf R}, 
and in {\tt Figures/} keep any figures you created 
for your answers.
% Similarly, keep any \LaTeX\ scripts for tables in the {\tt Tables/} folder.
% You may put any written text in the {\tt Text/} folder, 
% if not already included in a \texttt{tex} file in your {\tt Paper/} folder. 
Put anything else in the subdirectory {\tt Misc/}.
I should then be able to replicate all of your work simply by 
typing

\medskip
\begin{itemize}
\item {\tt \$ ./DoWork.sh}
\end{itemize}

\medskip\noindent
on the command line of a terminal window.

\bigskip
To provide you a template, which makes preparation easier for you and grading 
easier for me, I have placed sample \LaTeX\ and \textsf{R} code 
in the GitHub repository for the course: {\tt QMB6912S23},
under my GitHub username {\tt LeeMorinUCF};
pull this repository and use these files a framework within which to 
create the answers for this problem set.
Push the files to a folder on your GitHub repository
and I will pull your submissions to my computer for grading. 



\medskip
Be sure to support your calculations with descriptions 
of what you were trying to do (for example, in comments in your \textsf{R} code 
as well as in the \LaTeX\ explanations) because partial credit will be given.



\vfill
\centerline{\bf Due Date: Sunday, 22 January 2023, at 11:59 PM.}

\eject


%%%%%%%%%%%%%%%%%%%%%%%%%%%%%%%%%%%%%%%%%%%%%%%%%%%%%%%%%%%%

\end{document}

%%%%%%%%%%%%%%%%%%%%%%%%%%%%%%%%%%%%%%%%%%%%%%%%%%%%%%%%%%%%

